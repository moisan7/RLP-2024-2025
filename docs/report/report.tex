\documentclass[a4paper,12pt]{article}
\usepackage{fontspec} % Allows using modern fonts
\usepackage{xcolor} % For defining colors
\usepackage{graphicx} % For including images
\usepackage{titling} % For customizing the title page
\usepackage{amsmath} % For advanced mathematics
\usepackage{hyperref} % For links and references
\usepackage{unicode-math} % For using Unicode mathematics
\usepackage[margin=1in]{geometry} % For adjusting margins
\usepackage[english]{babel} % For the Spanish language (Note: The option is 'english', but the comment says Spanish. The content is in Spanish.)

\setmainfont{Aptos Serif} % Changes the main font to Georgia
\setmathfont{STIX Two Math} % Changes the math font to TeX Gyre Schola Math
\setmonofont{Cascadia Mono} % Changes the monospaced font to Cascadia Mono (Original comment said Consolas, but command is Cascadia Mono)

\begin{document}

% ============================
% Title Page
% ============================
\begin{titlepage}
    \centering
    \vspace*{2cm}

    {\Huge\bfseries COMPA - COMPRA  \\ \Large{Autonomous Shopping Cart} \par}

    \vspace{1cm}
    {\Large\bfseries Autonomous University of Barcelona (UAB)

    School of Computer Engineering - Robotics 2025 \par}

    \vspace{0.5cm}

    \includegraphics[width=0.5\textwidth]{Logo_uab.png} % UAB Logo (must be in the same folder as the document)

    \vspace{0.5cm}
    {\large Pol Tomé,

    Adrià Fernandez Mata,

    David Madueño Noguer,

    Tiago David Nunes Rodrigues,

    Moisés Sánchez Pin
     \par}

    \vspace{1cm}
    \includegraphics[width=0.5\textwidth]{compacompra_logo.jpg} % CompaCompra Logo (must be in the same folder as the document)

    \vfill
    {\Large \today} % This would render as the current date

\end{titlepage}

% ============================
% Table of Contents
% ============================
\tableofcontents % Translates to "Table of Contents" when rendered

% ============================
% Abstract of the document
% ============================
\begin{abstract}
Compa-Compra is an autonomous shopping assistant that guides and assists the store user to complete their shopping easily and simply. The user communicates with ComCom through an application where products are specified.
\textbf{Make your list and ComCom will guide you!}
\end{abstract}

% ============================
% Introduction of the document
% ============================
\section{Introduction}
The act of shopping is a daily activity, but it is not always efficient.
Whether due to the difficulty of finding products, congestion in the
aisles, or the need to compare options, navigating the supermarket
can become a tedious task. \textbf{Compa Compra} is an autonomous assistant
designed to improve the shopping experience for any user,
regardless of their profile, age, or specific needs.

Through a multimedia application, the customer enters their shopping
list, and the ComCom robot processes the best route within the establishment
using advanced localization and path planning techniques.
Thanks to the \textbf{D* Lite} algorithm and the use of \textbf{Bluetooth} beacons, \textbf{Compa Compra}
optimizes shopping time and facilitates efficient access to each product.

\section{Theoretical Framework and Contextualization}

The \textbf{CompaCompra} project is based on two key areas of robotics and autonomous navigation: \textbf{2D localization} using trilateration and route planning with \textbf{D* Lite}. These concepts allow the assistant to guide the user within the supermarket with precision and adaptability.

\subsection{2D Localization}
One of the main challenges of autonomous navigation is \textbf{indoor localization}. To determine \textbf{ComCom}'s position, 2D trilateration is used with four fixed Bluetooth beacons in the corners of the supermarket.

The method uses the measurement of the \textbf{Received Signal Strength Indicator (RSSI)} from each beacon to estimate the distance between the robot and the fixed references. However, the RSSI signal can be affected by interference, so \textbf{filtering and error correction} techniques, such as moving average and Kalman filter, are applied.

Once the estimated position \((x, y)\) is obtained, it is adjusted to the nearest supermarket grid cell, defined as a mesh of \textbf{0.5m² per node}. This process allows for a discrete representation of the navigation space and facilitates route calculation.

For more information on trilateration, consult the documentation:
\href{https://en.wikipedia.org/wiki/Trilateration}{Trilateration on Wikipedia}

\subsection{Path Finding Algorithm: D* Lite}
The optimal movement of the robot is managed by \textbf{D* Lite}, an efficient route-finding algorithm in dynamic environments. Based on the user's position and the products on the shopping list, the system calculates the ideal trajectory on the node grid.

Each node in the supermarket can be classified as:
\begin{itemize}
\item \textbf{Free:} traversable areas without obstacles.
\item \textbf{Occupied:} areas blocked by shelves, obstacles, or other customers' carts.
\end{itemize}

Initially, all known obstacles are considered, but the path is \textbf{dynamically recalculated} if \textbf{ComCom} detects new impediments during its journey. This allows the robot to adapt in real-time to changes in the environment.

For more information on \textbf{D* Lite}, consult the documentation:
\href{https://en.wikipedia.org/wiki/D*_Lite}{D* Lite on Wikipedia} % (Original link was to D*, I've pointed to D* Lite)

\section{Description of the CompaCompra System}

\end{document}