\documentclass[a4paper,12pt]{article}
\usepackage{fontspec} % Permite usar fuentes modernas
\usepackage{xcolor} % Para definir colores
\usepackage{graphicx} % Para incluir imágenes
\usepackage{titling} % Para personalizar la portada
\usepackage{amsmath} % Para matemáticas avanzadas
\usepackage{hyperref} % Para enlaces y referencias
\usepackage{unicode-math} % Para usar matemáticas Unicode
\usepackage[margin=1in]{geometry} % Para ajustar márgenes
\usepackage[spanish]{babel} % Para el idioma español

\setmainfont{Georgia} % Cambia la fuente principal a Georgia
\setmathfont{TeX Gyre Schola Math} % Cambia la fuente de matemáticas a TeX Gyre Schola Math
\setmonofont{Cascadia Mono} % Cambia la fuente monoespaciada a Consolas

\begin{document}

% ============================
% Portada
% ============================
\begin{titlepage}
    \centering
    \vspace*{2cm}

    {\Huge\bfseries \textsc{Compa - Compra} \\ Carrito de la compra autónomo  \par}

    \vspace{1.5cm}
    {\Large\bfseries Universitat Autònoma de Barcelona (UAB)
    
    Escola d'enginyeria informática - Robótica 2025 \par}

    \vspace{0.5cm}

    \includegraphics[width=0.5\textwidth]{Logo_uab.png} % Logo de la UAB (debe estar en la misma carpeta que el documento)

    \vspace{0.5cm}
    {\large Pol Tomé, 

    Adrià Fernandez Mata, 

    David Madueño Noguer,

    Tiago David Nunes Rodrigues, 

    Moisés Sánchez Pin
     \par}

    \vspace{2cm}
    \includegraphics[width=0.4\textwidth]{compacompra_logo.jpg} % Logo de la UAB (debe estar en la misma carpeta que el documento)

    \vfill
    {\Large \today}

\end{titlepage}

% ============================
% Tabla de contenido
% ============================
\tableofcontents

% ============================
% Abstract del documento
% ============================
\begin{abstract}
Compa Compra es un asistente de compra autónomo que se encarga de guíar y asistir al usuario del establecimiento para completar sus compras de forma fácil y sencilla. El usuario se comunica con ComCom mediante una aplicación en la que se especifican los productos.
\textbf{¡Haz la lista y ComCom te guía!}
\end{abstract}

% ============================
% Introducción del documento
% ============================  
\section{Introducción}
El acto de comprar es una actividad cotidiana, pero no siempre es eficiente. 
Ya sea por la dificultad de encontrar productos, la congestión en los 
pasillos o la necesidad de comparar opciones, recorrer el supermercado 
puede convertirse en una tarea tediosa. \textbf{ Compa Compra } es un asistente 
autónomo diseñado para mejorar la experiencia de compra de cualquier usuario, 
independientemente de su perfil, edad o necesidades específicas.  

A través de una aplicación multimedia, el cliente introduce su lista de 
compras, y el robot ComCom procesa la mejor ruta dentro del establecimiento 
usando técnicas avanzadas de localización y planificación de trayectos. 
Gracias al algoritmo *D* Lite y el uso de balizas Bluetooth, \textbf{ Compa Compra } 
optimiza el tiempo de compra y facilita el acceso eficiente a cada producto.  

\section{Marco Teórico y Contextualización}

El proyecto \textbf{CompaCompra} se fundamenta en dos áreas clave de la robótica y la navegación autónoma: la \textbf{localización 2D} mediante trilateración y la planificación de rutas con \textbf{D* Lite}. Estos conceptos permiten que el asistente guíe al usuario dentro del supermercado con precisión y adaptabilidad.

\subsection{Localización 2D}
Uno de los principales retos de la navegación autónoma es la \textbf{localización en interiores}. Para determinar la posición de \textbf{ComCom}, se emplea trilateración 2D con cuatro balizas Bluetooth fijas en las esquinas del supermercado.

El método utiliza la medición de la \textbf{intensidad de señal recibida (RSSI)} de cada baliza para estimar la distancia entre el robot y las referencias fijas. Sin embargo, la señal RSSI puede verse afectada por interferencias, por lo que se aplican técnicas de \textbf{filtrado y corrección de errores}, como la media móvil y el filtro de Kalman.

Una vez obtenida la posición estimada \((x, y)\), se ajusta a la cuadrícula más cercana del supermercado, definida como una malla de \textbf{0.5m² por nodo}. Este proceso permite una representación discreta del espacio de navegación y facilita el cálculo de rutas.

Para más información sobre trilateración, consultar la documentación:
\href{https://es.wikipedia.org/wiki/Trilateraci%C3%B3n}{Trilateración en Wikipedia}

\subsection{Algoritmo de Path Finding: D* Lite}
El desplazamiento óptimo del robot se gestiona mediante \textbf{D* Lite}, un algoritmo eficiente de búsqueda de rutas en entornos dinámicos. A partir de la posición del usuario y los productos en la lista de compra, el sistema calcula la trayectoria ideal en la malla de nodos.

Cada nodo del supermercado puede clasificarse en:
\begin{itemize}
\item \textbf{Libre:} zonas transitables sin obstáculos.
\item \textbf{Ocupado:} áreas bloqueadas por estanterías, obstáculos o carros de otros clientes.
\end{itemize}

Inicialmente, se consideran todos los obstáculos conocidos, pero el camino se \textbf{recalcula dinámicamente} si \textbf{ComCom} detecta nuevos impedimentos durante su recorrido. Esto permite que el robot se adapte en tiempo real a los cambios en el entorno.

Para más información sobre \textbf{D* Lite}, consultar la documentación:
\href{https://es.wikipedia.org/wiki/D*#D*_Lite}{D* Lite en Wikipedia}

\section{Descripción del Sistema CompaCompra}

\end{document}
